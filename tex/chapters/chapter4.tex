\cleardoublepage

\chapter{Dynamic Mode Decomposition}
\label{chapter:DMD-PM}

Before introducing how to use dynamic mode decomposition to accelerate the power and flattened power methods, we will first review the details of DMD.
As mentioned in Chapter 1, the details of DMD are different based on the application.
However, most of these varieties share a familiar, straightforward frame. 
Here, the most widely-used variant (called ``standard DMD'' here) will be shown as an example to represent the algorithm.

To start, first consider the generic, dynamic problem defined by 
\begin{equation}
  \frac{d{\mathbf{x}}}{dt}=\mathbf{f}(\mathbf{x},t) \, ,
  \label{eq:dynamic_problem}
\end{equation}
where $\mathbf{x} \in \mathbb{R}^{n}$ is the $n$-dimensional state vector at time $t$. 
With sufficiently small steps in time, the evolution of  $\mathbf{x}$ can be well approximated by a relationship of the form 
\begin{equation}
 \frac{d{\mathbf{x}}(t)}{dt}=\mathcal{A}\mathbf{x} \, ,
 \label{eq:linearized_model}
\end{equation}
where the evolution operator $\mathcal{A}$ may be unknown and can be considered a ``black-box'' system.
However, one can obtain the system state $\mathbf{x}_n$ at different times, which are then stacked as the past and future snapshot matrices $ \mathbf{X}_0$ and $ \mathbf{X}_1$, i.e., 
\begin{equation}
\label{eq:past_data}
\mathbf{X_0}=\left[ \mathbf{x}_0, \mathbf{x}_1, \ldots, \mathbf{x}_{m-1} \right] \, ,
\end{equation}
and
\begin{equation}
\label{eq:future_data}
\mathbf{X_1}=\left[ \mathbf{x}_1, \mathbf{x}_2, \ldots, \mathbf{x}_{m} \right] \, .
\end{equation}

Suppose $\mathbf{A}$ is the discrete-time approximation of the mapping operator $\mathcal{A}$:
\begin{equation}
\label{eq:A}
\mathbf{A} = e^{\mathcal{A} \Delta t} \, .
\end{equation}
Then,
\begin{equation}
\label{eq:x0x1}
\mathbf{x}_{k+1} = \mathbf{A} \mathbf{x}_{k}, \quad \quad k = 0,1,...,\, .
\end{equation}
In general, the approximate operator $\mathbf{A}$ not reproduce the $\mathbf{x}_i$ exactly, but a ``best'' approximation can be formed in a least-squares or minimum-norm sense by solving
\begin{equation}
\label{eq:least}
\mathbf{A} =  \underset{\mathbf{A}}{argmin} ||\mathbf{X}_1 - \mathbf{A} \mathbf{X}_0||_F \, .
\end{equation}
Thus, the best-fit operator $\mathbf{A}$ is formally given by 
\begin{equation}
\label{eq:fullA}
A = \mathbf{X}_1 \mathbf{X}_0^{\dagger} \, ,
\end{equation}
where $\mathbf{X}_0^{\dagger}$ is the Moore-Penrose generalized inverse of $\mathbf{X}_1$.
It is possible (and typical) to use SVD factorization to find the inverse of $\mathbf{X}_1$ by
\begin{equation}
\label{eq:svd}
\mathbf{X}_0 = \mathbf{U} \bm{\Sigma} \mathbf{V}^{*} \rightarrow \mathbf{X}_0^{\dagger} = \mathbf{V} \bm{\Sigma}^{-1} \mathbf{U}^* \, ,
\end{equation}
where $\mathbf{U} \in \mathbb{C}^{m\times n}$, $\mathbf{V} \in \mathbb{C}^{n\times n}$, $\bm{\Sigma} \in \mathbb{C}^{n\times n}$, and $*$ indicate the conjugate transposes. 

However, considering that the number of unknowns in this matrix is often large during numerical simulations, the matrix $\mathbf{A}$ is not computed explicitly.
A low-rank approximation of the original dynamic system $\mathbf{\tilde{A}}$ is formed, i.e.,
\begin{equation}
\label{eq:reduced_dmd_1}
\mathbf{\tilde{A}} =  \mathbf{U_r^* A U_r} \, .
\end{equation}
Then using \EQ{eq:fullA}, the reduced order $\mathbf{\tilde{A}}$ is defined by
\begin{equation}
\label{eq:reduced_dmd_2}
\mathbf{\tilde{A}} =  \mathbf{U_r^*} \mathbf{X_1} \mathbf{V} \bm{\Sigma}^{-1} \, .
\end{equation}
Now extract the $r$ largest eigenvalue and corresponding eigenvectors from $\mathbf{\tilde{A}}$ as the DMD modes $\boldsymbol{\Phi}$, which can be treated as the leading $r$ eigenvectors of $\mathbf{A}$.
Note that the solution of \EQ{eq:linearized_model} is 
\begin{equation}
\mathbf{x}(t) = e^{\mathcal{A}t}\mathbf{x}(0) \, ,   
\end{equation}
and $\mathbf{A}$ is a discrete-time approximation of $e^{\mathcal{A}\Delta}$, which can be applied to the initial condition using the matrix exponential to compute the solution at a particular time. 
Moreover, the discrete eigenvalues $\lambda_i$ of $\mathbf{A}$ can be used to compute the continuous eigenvalues $\omega_i= \text{log}(\lambda_i)/\Delta t$.
Subsequently, the state can be reconstructed at any time $t$ by a given initial condition 
\begin{equation}
\label{eq:dmd_predict}
\vec{x}^{DMD}(t) \triangleq \sum_{i=1}^{r} \vec{\phi}_i e^{\omega_it} b_i \, ,
\end{equation}
where $\mathbf{b}=\boldsymbol{\Phi}^{\dag} \mathbf{x}_{0}$. 

The general DMD scheme is summarized as
\begin{enumerate}
\item Compute SVD decomposition of the forward snapshots matrix, i.e., $ \mathbf{X_0} = \mathbf{U_r} \boldsymbol{\Sigma_r} \mathbf{V_r^{*}}$, where r indicates the rank of matrix.
\item Compute $\mathbf{\tilde{A}}=\mathbf{U_r^{*}AU_r}=\mathbf{U_r^{*}X_1}\mathbf{V_r}\boldsymbol{\Sigma}_{\mathbf{r}}^{-1}$, where $\mathbf{A}$ and $\mathbf{\tilde{A}}$ are similar matrices.
\item Compute the eigendecomposition $\mathbf{\tilde{A} \tilde{W}}=\boldsymbol{\Lambda}\mathbf{\tilde{W}}$.
\item Calculate the DMD modes as ${\boldsymbol{\Phi}}={\mathbf{X_1V_r}}\boldsymbol{\Sigma}_\mathbf{r}^{-1}{\mathbf{\tilde{W}}}$.
\item Predict the response by $\vec{x}^{DMD}(t) \approx \sum_{i=1}^{r} \vec{\phi}_i e^{\omega_it} b_i = \boldsymbol{\Phi}{\mathbf{diag}}(e^{\vec{\omega}t})\vec{b}$, where $\mathbf{b}=\boldsymbol{\Phi}^{\dag} \mathbf{x}_{0}$.
\end{enumerate}


