% +--------------------------------------------------------------------+
% | Sample Chapter 2
% +--------------------------------------------------------------------+

\cleardoublepage

\chapter{MultiGroup Transport and Diffusion Equation}
\label{makereference2}
Since it is impossible to simulate the location, moving direction, and moving speed of neutrons, neutron transport equation is employed to model the neutron particle kinetics and neutron population in the systems.
Neutron transport is a extension of Boltzmann transport equation, which describes the statistical behaviour of particles in the dynamic thermal systems.

Fix source and Eigenvalue

\section{Numerical approach}
Both fixed-source and criticality calculations can be solved using deterministic methods or stochastic methods. In deterministic methods the transport equation (or an approximation of it, such as diffusion theory) is solved as a differential equation. In stochastic methods such as Monte Carlo discrete particle histories are tracked and averaged in a random walk directed by measured interaction probabilities. Deterministic methods usually involve multi-group approaches while Monte Carlo can work with multi-group and continuous energy cross-section libraries. Multi-group calculations are usually iterative, because the group constants are calculated using flux-energy profiles, which are determined as the result of the neutron transport calculation.



After these assumptions and the further assumption of isotropic scattering, the neutron transport equation reduces to
\begin{equation}
\begin{split}
  \bm{\hat{\Omega}} & \cdot \nabla \psi_g(\vec{r},\bm{\hat{\Omega}}) +
    \Sigma_{t g}(\vec{r}) \psi_{g}(\vec{r},\bm{\hat{\Omega}}) = \\
   & \frac{1}{4\pi} \sum\limits^{N_g}_{g'=1} \Sigma_{s g g'}(\vec{r}) \phi_{g'}(\vec{r}) +
    \frac{\chi_g}{4\pi k} \sum\limits^{N_g}_{g'=1} \nu\Sigma_{fg'}(\vec{r}) \phi_{g'}(\vec{r}) \, ,
\end{split}
\label{eq:transport}
\end{equation}
where $\psi_g$ represents the angular flux and $\phi_g$ is the 
group-dependent scalar flux in certain energy group $g$.  
In addition, $\Sigma_{t,g}$, $\Sigma_{s,g\prime g}$, and $\Sigma_{f,g}$ represent the group dependent cross sections for total, inscattering, and fission respectively.  
$\chi_g$ is the fission spectrum, $\nu$ is the average number of neutrons emitted per fission, and the $k$-eigenvalue (or ``multiplication factor'') represents the balance of neutron gains (by fission) to losses (by absorption and leakage).  A reactor is ``critical,'' i.e., the neutron population is steady, when $k=1$.

\section{Operator Notation}
The transport equation may be written more compactly in operator notation as
\begin{equation}
 \mathbf{(I - DL^{-1}MS)} \mathbf{\phi} = \frac{1}{k} \mathbf{DL^{-1}MF} \mathbf{\phi}  \, ,
 \label{eq:keig}
\end{equation}
where $\mathbf{\phi} $ describes the scalar flux, $\mathbf{L}$ represents loss operator,  $\mathbf{F}$ represents fission operator, $\mathbf{S}$ represents the scattering operator, $\mathbf{M}$ and $\mathbf{D}$ represent moment-to-discrete and discrete-to-moment operators respectively, and the eigenvalue $k$ represents the ratio of gains to losses. 