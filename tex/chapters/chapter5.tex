\cleardoublepage

\chapter{The DMD-FPM(n) Method}
\label{chapter:DMD-FPM(n)}

As mentioned in \CHAPTER{chapter:PM}, the flattened power method is a more efficient approach for solving the multigroup neutron transport equation.
We also discussed a restarted, DMD-accelerated power method scheme in the last chapter, which suggest that there is potential to develop this algorithm to fit the flattened power method.
This chapter contains a complete description of an accelerated flattened power method using DMD, which has been summarized in a transaction by the author of this thesis \cite{xu_acceleration}. 

\section{Methodology}
The basic framework of DMD-FPM($n$) is similar to DMD-PM($n$). The main difference is that the snapshots are now generated by the flattened power method.
In short, a set of flattened power iterations are performed, then, DMD uses the snapshots to correct the dominant mode and eigenvalue, which is used to continue power iterations.
The process can be repeated until the results converge, which leads to a {\it restarted} DMD-FPM (or DMD-FPM($n$)).

\subsection{Aitken Extrapolation}
Note that not only is the eigenvector required at every restarting point but also the corresponding eigenvalue.
The updated eigenvalue in the DMD-PM($n$) algorithm is equal to the norm of the DMD predicted eigenmodes. 
One significant drawback of the flattened power method is that we cannot compute the corresponding eigenvalue by the DMD response, because the eigenvalue has already been applied to the snapshots of neutron flux.
In other words, there is no easy way to find the eigenvalue $k$ associated with the predicted $\mathbf{x}$, which is projected forward in ``time".
Therefore, there would be an inherent mismatch between the accelerated dominant eigenvector and its eigenvalue.

Many mathematical approaches were tested to extrapolate an eigenvalue to match the DMD eigenvector. 
A first attempt used the last computed eigenvalue, i.e., an eigenvalue that may be the equivalent of tens or hundreds of flattened power iterations in the ``past."
However, the error caused by that choice of eigenvalue tended to reduce the improvement of the DMD extrapolation significantly, which erased all the improvement from DMD sometimes.
Another attempt was to insert the eigenvalue as the first element in the snapshot, which did not work either.
The reason might be that we modified the standard DMD algorithm, and only dominant modes were used.
Some other failures include linear and polynomial extrapolation. 
In order to predict a more appropriate eigenvalue, Aitken extrapolation was employed \cite{aitken_1927} as
\begin{equation}
k_{aitken} = k_{i-2} - \frac{(k_{i-1}-k_{i-2})^2}{(k_i - 2k_{i-1} + k_{i-2})}\, ,
\end{equation}
where $k_i$ is the eigenvalue from the ith iteration of the flattened power iteration.
Although Aitken extrapolation does not eliminate the error from eigenvector/eigenvalue mismatch completely, significant improvement in numerical tests were observed.

The procedure for applying DMD to the flattened power iteration with Aitken extrapolation is summarized as follows.

\begin{enumerate}
 \item Assume $k_{(0)}$, $\mathbf{x}_{(0)}$ and normalize.
 \item Perform $n$ flattened operator applications (\EQ{eq:flatten}) to produce $\mathbf{X}_0$ and $\mathbf{X}_1$.
 \item Compute the DMD modes and frequencies using a rank-$r$, truncated  SVD (i.e., $r < n$).
 \item Apply equation $\mathbf{x}_{0}=\frac{b_0 \vec{\phi}^{}_0}{ ||b_0 \vec{\phi}^{}_0||}$ to estimate $\mathbf{x}^{(\infty)}=\mathbf{x}(\infty)$, i.e., estimate the steady-state, dominant mode after an equivalent of $\infty$ power iterations.
 \item Update $k_{(0)}$ by Aitken extrapolation.
 \item Repeat Steps 1 through 5 until converged.
\end{enumerate}

Both DMD-PM($n$) and DMD-FPM($n$) are tested to verify the performance to accelerate solving neutron transport/diffusion problem. The testing cases and numerical analysis are presented in the following chapter. 


