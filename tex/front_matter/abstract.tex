% +--------------------------------------------------------------------+
% | Abstract Page
% +--------------------------------------------------------------------+

\pagestyle{empty}
%\vspace{1cm}
\setlength{\baselineskip}{0.8cm}

%\indent

% +--------------------------------------------------------------------+
% | Enter the text of your abstract below.  There is no limit on the
% | number of words in your abstract.
% +--------------------------------------------------------------------+

An algorithm based on dynamic mode decomposition (DMD) is presented for acceleration of the power method (PM) and flattened power method (FPM) that takes advantage of prediction from a restarted DMD process to correct an unconverged solution.
The power method is a simple iterative scheme for determining the dominant eigenmode, and its variants, such as flattened power method, have long been used to solve the k-eigenvalue problem in reactor analysis.
DMD is a data driven technique that extracts dynamics information from time-series data with which a reduced-order surrogate model can be constructed.
DMD-accelerated PM (DMD-PM) and DMD-accelerated FPM (DMD-FPM) generate ``snapshots'' from a few iterations and extrapolate space in ``fictitious time'' to produce a more accurate estimate of the dominant mode.  
This process is repeated until the solution is converged to within a suitable tolerance.
To illustrate the performance of both two schemes, a 1-D test problem designed to resemble a boiling water reactor (BWR) and the well-studied 2-D C5G7 benchmark were analyzed.
Compared to the PM without acceleration, these tests have demonstrated that DMD-PM and DMD-FPM method can reduce the number of iterations significantly.


