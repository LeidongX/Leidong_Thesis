% +--------------------------------------------------------------------+
% | Sample Chapter 2
% +--------------------------------------------------------------------+

\cleardoublepage

% +--------------------------------------------------------------------+
% | Replace "This is Chapter 2" below with the title of your chapter.
% | LaTeX will automatically number the chapters.                      
% +--------------------------------------------------------------------+

\chapter{Conclusions and Future Work}
\label{chapter:conclusion}
\section{Summary}
In this section, we summarize the main findings in this thesis.
First, the principal objective of this thesis was to accurately estimate fundamental eigenmodes by DMD to accelerate the power method and flattened power method for multigroup neutron transport/diffusion problems.
Some successful cases using DMD on producing reduced order surrogate models or accelerating iterative methods were introduced in \CHAPTER{chapter:intro}.
Some backgrounds, including the power method, the flattened power method and the multigroup transport/diffusion theory, were explored in \CHAPTER{chapter:multigroup} and \CHAPTER{chapter:PM}, in order to to understand the potential circumstances for the acceleration schemes.

Although DMD is a useful tool for extracting information from data, often it can only be applied on time-dependent dynamic systems. 
In \CHAPTER{chapter:DMD-PM}, we have presented a new definition for identifying fundamental eigenvector by a improved DMD algorithm, which avoids computing all the DMD modes.   
The approach DMD-PM($n$) allows us to correct current results by a accurate estimation of the steady solution. 
This restarted version of this scheme can be applied repeatedly to accelerate the power method.
We also explored a similar variety DMD-FPM($n$) based this scheme to improve the efficiency of the flattened power method, where the key is to take advantage of Aitken extrapolation to match the eigenvalues.

Three test problems were conducted, which include (1) a 2-D, IAEA diffusion benchmark, (2) a 1-D, 70-pin BWR core model, and (3) the 2-D C5G7 benchmark.
Through numerical examples, we demonstrated that both acceleration schemes provide promising speedup compared to unaccelerated schemes. 
The DMD-PM(n) method used only 25\% full power iterations for the 2-D, IAEA diffusion problem, As can be expected, DMD-FPM(n) provided approximately a 5x$-$10x speedup for the cases studied.

\section{Future Research}
In this section, we describe the future direction and substantial value of using these methods in other areas.
While these results are promising, the performance of both schemes are not expected to outperform some other popular methods, such as the generalized Davidson method\cite{hamilton2011numerical} and coarse-mesh finite difference\cite{smith_1983}.
In reactor analysis, the use of stochastic methods, such as the famous Monte Carlo simulations, is widespread.
Following this thesis, application of DMD-PM($n$) and DMD-FPM($n$) may be able to accelerate Monte Carlo eigenvalue problems for convergence by regressing the distribution tendency of neutron population from DMD modes.
In this way, only a small size of neutron populations are sufficient to generate snapshots and extract information, which might be comparable to the other acceleration methods.

Although Aitken extrapolation could estimate the eigenvalues corresponding to DMD responses, errors still exist at almost every restart point, and, therefore, reducing the desired accuracy. 
More Work can also be done to study to match the eigenvalues, which can improve the performance while a great amount of restarted process is required in the large scale systems.



